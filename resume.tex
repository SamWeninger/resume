\documentclass[10pt, letterpaper]{article}

% Packages:
\usepackage[
    ignoreheadfoot, % set margins without considering header and footer
    top=2 cm, % seperation between body and page edge from the top
    bottom=2 cm, % seperation between body and page edge from the bottom
    left=2 cm, % seperation between body and page edge from the left
    right=2 cm, % seperation between body and page edge from the right
    footskip=1.0 cm, % seperation between body and page edge from the footer
    % showframe % for debugging 
]{geometry} % for adjusting page geometry
\usepackage[explicit]{titlesec} % for customizing section titles
\usepackage{tabularx} % for making tables with fixed width columns
\usepackage{array} % tabularx requires this
\usepackage[dvipsnames]{xcolor} % for coloring text
\usepackage{enumitem} % for customizing lists
\usepackage{fontawesome5} % for using icons
\usepackage{amsmath} % for math

% Import user configuration
%%% USER CONFIGURATION FILE %%%
% Edit this file to customize your resume without touching the main template

% User Information Variables
\newcommand{\firstName}{Samuel}
\newcommand{\lastName}{Weninger}
\newcommand{\name}{\firstName\ \lastName}
\newcommand{\profession}{Software Engineer}

% Contact Information
\newcommand{\phone}{(1)206-679-2962}
\newcommand{\email}{sweninger99@gmail.com}

% Social Links - Full URLs
\newcommand{\githuburl}{https://github.com/SamWeninger}
\newcommand{\linkedinurl}{https://www.linkedin.com/in/samuel-weninger/}
\newcommand{\websiteurl}{https://www.samweninger.com/}

% Social Links - Display Text
\newcommand{\githubtext}{SamWeninger}
\newcommand{\linkedintext}{samuel-weninger}
\newcommand{\websitetext}{samweninger.com}

% Theme Colors - Edit the RGB values to change the theme color
\definecolor{primaryColor}{RGB}{33, 113, 187} % Blue
\definecolor{secondaryColor}{RGB}{150, 150, 150} % Light Gray
\definecolor{tertiaryColor}{RGB}{80, 80, 80} % Dark Gray
\definecolor{quaternaryColor}{RGB}{60, 60, 60} % Darker Gray

% Document Settings
\newcommand{\footertext}{\name} % Text that appears in the footer 

\usepackage[
    pdftitle={\name's Resume},
    pdfauthor={\name},
    pdfcreator={LaTeX},
    colorlinks=true,
    urlcolor=primaryColor
]{hyperref} % for links, metadata and bookmarks
\usepackage[pscoord]{eso-pic} % for floating text on the page
\usepackage{calc} % for calculating lengths
\usepackage{bookmark} % for bookmarks
\usepackage{lastpage} % for getting the total number of pages
\usepackage{changepage} % for one column entries (adjustwidth environment)
\usepackage{paracol} % for two and three column entries
\usepackage{ifthen} % for conditional statements
\usepackage{needspace} % for avoiding page brake right after the section title
\usepackage{iftex} % check if engine is pdflatex, xetex or luatex

% Ensure that generate pdf is machine readable/ATS parsable:
\ifPDFTeX
    \input{glyphtounicode}
    \pdfgentounicode=1
    \usepackage[T1]{fontenc}
    \usepackage[utf8]{inputenc}
    \usepackage{lmodern}
\fi

\usepackage[default, type1]{sourcesanspro} 

% Some settings:
\AtBeginEnvironment{adjustwidth}{\partopsep0pt} % remove space before adjustwidth environment
\pagestyle{empty} % no header or footer
\setcounter{secnumdepth}{0} % no section numbering
\setlength{\parindent}{0pt} % no indentation
\setlength{\topskip}{0pt} % no top skip
\setlength{\columnsep}{0.15cm} % set column seperation
\makeatletter
\let\ps@customFooterStyle\ps@plain % Copy the plain style to customFooterStyle
\patchcmd{\ps@customFooterStyle}{\thepage}{
    \color{gray}\textit{\small \footertext}
}{}{} % replace number by desired string
\makeatother
\pagestyle{customFooterStyle}

\titleformat{\section}{
    % avoid page braking right after the section title
    \needspace{4\baselineskip}
    % make the font size of the section title large and color it with the primary color
    \Large\color{primaryColor}
}{
}{
}{
    % print bold title, give 0.15 cm space and draw a line of 0.8 pt thickness
    % from the end of the title to the end of the body
    \textbf{#1}\hspace{0.15cm}\titlerule[0.8pt]\hspace{-0.1cm}
}[] % section title formatting

\titlespacing{\section}{
    % left space:
    -1pt
}{
    % top space:
    0.3 cm
}{
    % bottom space:
    0.2 cm
} % section title spacing

% \renewcommand\labelitemi{$\vcenter{\hbox{\small$\bullet$}}$} % custom bullet points
\newenvironment{highlights}{
    \begin{itemize}[
        topsep=0.10 cm,
        parsep=0.10 cm,
        partopsep=0pt,
        itemsep=0pt,
        leftmargin=0.4 cm + 10pt
    ]
}{
    \end{itemize}
} % new environment for highlights

\newenvironment{highlightsforbulletentries}{
    \begin{itemize}[
        topsep=0.10 cm,
        parsep=0.10 cm,
        partopsep=0pt,
        itemsep=0pt,
        leftmargin=10pt
    ]
}{
    \end{itemize}
} % new environment for highlights for bullet entries


\newenvironment{onecolentry}{
    \begin{adjustwidth}{
        0.2 cm + 0.00001 cm
    }{
        0.2 cm + 0.00001 cm
    }
}{
    \end{adjustwidth}
} % new environment for one column entries

\newenvironment{twocolentry}[2][]{
    \onecolentry
    \def\secondColumn{#2}
    \setcolumnwidth{\fill, 4.5 cm}
    \begin{paracol}{2}
}{
    \switchcolumn \raggedleft \secondColumn
    \end{paracol}
    \endonecolentry
} % new environment for two column entries

\newenvironment{threecolentry}[3][]{
    \onecolentry
    \def\thirdColumn{#3}
    \setcolumnwidth{1 cm, \fill, 4.5 cm}
    \begin{paracol}{3}
    {\raggedright #2} \switchcolumn
}{
    \switchcolumn \raggedleft \thirdColumn
    \end{paracol}
    \endonecolentry
} % new environment for three column entries

% save the original href command in a new command:
\let\hrefWithoutArrow\href

% new command for external links:
\renewcommand{\href}[2]{\hrefWithoutArrow{#1}{\ifthenelse{\equal{#2}{}}{ }{#2 }\raisebox{.15ex}{\footnotesize \faExternalLink*}}}

% Custom commands for resume entries
\newcommand{\resumeSubHeadingListStart}{\begin{itemize}[leftmargin=*]}
\newcommand{\resumeSubHeadingListEnd}{\end{itemize}}
\newcommand{\resumeItemListStart}{\begin{itemize}[leftmargin=*]}
\newcommand{\resumeItemListEnd}{\end{itemize}}

\newcommand{\resumeSubheading}[4]{
  \item
    \begin{tabular*}{0.97\textwidth}[t]{l@{\extracolsep{\fill}}r}
      \textbf{#1} & #2 \\
      \textit{#3} & \textit{#4} \\
    \end{tabular*}\vspace{-5pt}
}

\newcommand{\resumeProjectHeading}[2]{
  \item
    \begin{tabular*}{0.97\textwidth}[t]{l@{\extracolsep{\fill}}r}
      #1 & #2 \\
    \end{tabular*}\vspace{-5pt}
}

\newcommand{\resumeItem}[1]{
  \item{#1}
}

\begin{document}

% Include the header section
% Header section
\begin{center}
    \Huge \firstName\ \textbf{\lastName} \\ \vspace{3pt}
    \small
    \faMobile \hspace{1pt} \href{tel:+\phone}{\phone}
    \hspace{3pt} $|$ \hspace{3pt}
    \faEnvelope \hspace{1pt} \href{mailto:\email}{\email}
    \hspace{3pt} $|$ \hspace{3pt}
    \faGithub \hspace{1pt} \href{\githuburl}{\githubtext}
    \hspace{3pt} $|$ \hspace{3pt}
    \faLinkedin \hspace{1pt} \href{\linkedinurl}{\linkedintext}
    \hspace{3pt} $|$ \hspace{3pt}
    \faGlobe \hspace{1pt} \href{\websiteurl}{\websitetext}
\end{center} 

% Include the experience section
% Experience section
\section{Experience}
\resumeSubHeadingListStart
    \item
        \textbf{Microsoft} \hfill \textcolor{primaryColor}{Redmond, WA} \\
        \textcolor{tertiaryColor}{\textit{Software Engineer}} \hfill \textcolor{secondaryColor}{\textit{Sept 2022 - Aug 2024}}
    \resumeItemListStart
        \resumeItem{Implemented language backfill logic and enhanced language extraction for \textbf{+2 Billion} Urls indexed by Bing, exposing rich content for \textbf{+100 million} businesses in \textbf{+150} different markets on Bing. (\textbf{C\#})}
        \resumeItem{Improved the coverage of Bing Amenities for \textbf{+6 Million} new businesses (\textbf{+55\%} feature coverage improvement). Enhanced Bing Amenities quality by \textbf{+30\%} leveraging \textbf{LLMs}, benefiting \textbf{+14M} businesses and reducing frequency of user-reported Bugs by \textbf{+80\%}. (\textbf{C\#, Python})}
        \resumeItem{Engineered automated pipeline (\textbf{ADF}) for discovering new domains for Bing; onboarded \textbf{+500} new providers, enriching data for \textbf{+2.7B} businesses. Managed vendor team to develop models for \textbf{HTML} content extraction from new domains, integrating content into Bing. (\textbf{SQL})}
        \resumeItem{Developed real-time monitoring dashboard for Bing rich data with routed alerts for quick, \textbf{on-call} response to data outages \& bugs. (\textbf{HTML})}
    \resumeItemListEnd

    \item
        \textbf{Huawei Technologies Canada} \hfill \textcolor{primaryColor}{Markham, ON (Remote)} \\
        \textcolor{tertiaryColor}{\textit{Software Engineer Intern}} \hfill \textcolor{secondaryColor}{\textit{May 2020 - Aug 2021}}
    \resumeItemListStart
        \resumeItem{Collaborated with a team of architects \& engineers to develop (\textbf{C/C++}) the infrastructure for a distributed database leveraged in \textbf{5G networks}.}
        \resumeItem{Optimized storage processing logic \& improved transactional procedures in a distributed database system (\textbf{C++, SQL}).}
    \resumeItemListEnd
\resumeSubHeadingListEnd 

% Include the skills section
% Skills section
\section{Skills}
\vspace{-0.1cm}
\begin{tabular*}{\textwidth}{@{}r@{\hspace{.4cm}}l@{}}
    \textbf{Software} & \textcolor{quaternaryColor}{Python $\cdot$ JavaScript $\cdot$ C\# $\cdot$ C++ $\cdot$ SQL $\cdot$ HTML $\cdot$ CSS} \\
    \textbf{Frameworks} & \textcolor{quaternaryColor}{React.js $\cdot$ Node.js $\cdot$ Flask $\cdot$ FastAPI $\cdot$ .NET} \\
    \textbf{Technical} & \textcolor{quaternaryColor}{Linux $\cdot$ Bash $\cdot$ Git $\cdot$ AWS $\cdot$ Docker $\cdot$ LLMs $\cdot$ PostgreSQL $\cdot$ MongoDB} \\
\end{tabular*}

% Include the projects section
% Projects section
\section{Projects}
\resumeSubHeadingListStart
    \item
        \textbf{Pebbles} \hfill \textcolor{primaryColor}{Toronto, ON} \\
        University Design Team Recommendation App \hfill \textcolor{secondaryColor}{Jun 2021 - Sept 2022}
    \resumeItemListStart
        \resumeItem{Created recommendations application to cluster students \& professors into engineering design teams based on internal \& external signals/interests}
        \resumeItem{Full stack application with database (\textbf{AWS DynamoDB}), backend API (\textbf{Python} + \textbf{AWS}) \& frontend application (\textbf{TypeScript})}
    \resumeItemListEnd

    \item
        \textbf{Course Finder} \hfill \textcolor{primaryColor}{Toronto, ON} \\
        Course Planning Web App \hfill \textcolor{secondaryColor}{Sept 2021 - May 2022}
    \resumeItemListStart
        \resumeItem{Created full stack application designed for course registration \& selection at University of Toronto, with user profile \& authentication}
        \resumeItem{Integrated database (\textbf{PostgreSQL}), backend API (\textbf{Python Flask} + \textbf{SQLAlchemy} ORMs) \& frontend application (\textbf{React.js})}
    \resumeItemListEnd

    % \resumeSubheading
    % {Jane Street Electronic Trading Challenge \faExternalLink*}{Toronto, ON}
    % {Finalist: Placed 2/100+ Undergraduate \& Graduate Student Participants}{June 2019}
    % \resumeItemListStart
    %     \resumeItem{Developed (\textbf{C++}) \& deployed (\textbf{EC2}) trading algorithm to realize arbitrage opportunities in ETFs/ADRs \& maximize profit in a simulated market}
    % \resumeItemListEnd
\resumeSubHeadingListEnd 

% Include the education section
% Education section
\section{Education}
\resumeSubHeadingListStart
    \item
        \textbf{B.S. (Honours) in Computer Engineering} \hfill \textcolor{primaryColor}{Toronto, ON} \\
        University of Toronto \hfill \textcolor{secondaryColor}{Sept 2017 - May 2022}
    \resumeItemListStart
        \resumeItem{3.72 GPA}
    \resumeItemListEnd
\resumeSubHeadingListEnd 

\end{document}